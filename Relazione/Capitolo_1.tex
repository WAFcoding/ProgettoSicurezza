\def \ti{\textit}
\def \bf{\textbf}

\chapter{Introduzione}
	\label{cap:intro}
	
\section{Scopo del progetto}
	\label{sec:scopo}

Il progetto sviluppato in questi mesi per il corso di ``Sicurezza Informatica e Internet'' si propone come scopo quello di realizzare un'applicazione che permetta a due o più utenti di scambiarsi documenti cartacei in modo sicuro. In particolare questo significa che l'utente che utilizza tale sistema deve essere in grado di compilare un foglio con informazioni sensibili e di mandarlo ad un insieme di utenti facendo in modo che nessun altro al di fuori dei destinatari possa leggerne il contenuto. Viceversa i destinatari devono essere altrettanto sicuri del fatto che durante il trasferimento nessuno possa alterare il documento senza che chi lo riceve se ne accorga.
Quindi, riassumendo, il sistema dovrebbe permettere di:
\begin{itemize}
	\item compilare un documento contenente qualsiasi tipo di informazione;
	\item cifrare in tutto o in parte il documento appena scritto;
	\item allegare le informazioni cifrate al documento stesso così che il o i destinatari possano leggerlo;
	\item firmare il documento in modo tale che il mittente non possa ripudiarne la provenienza e i destinatari possano accorgersi di eventuali manomissioni durante il trasferimento;
	\item specificare diversi livelli di privilegio gerarchici per la lettura delle informazioni.
\end{itemize}
Quest'ultimo punto in particolare rappresenta un requisito importante. Può capitare, infatti, soprattutto in ambienti amministrativi, di voler pubblicare un certo documento pur mantenendo segrete alcune informazioni sensibili. In questo caso la soluzione più ovvia potrebbe essere quella di pubblicare versioni diverse dello stesso documento oscurando opportunamente sulle diverse copie le informazioni che si vogliono tenere segrete a ciascun gruppo. Questo processo può risultare lungo, macchinoso e poco sicuro: è possibile, infatti, che qualche copia possa finire al destinatario sbagliato se non si prendono precauzioni.

\section{Stato dell'arte}
	\label{sec:statoarte}

\section{Tecnologie Utilizzate}
	\label{sec:tecnologie}