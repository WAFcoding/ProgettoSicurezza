\def \ti{\textit}
\def \bf{\textbf}

\chapter{Conclusioni}
	\label{cap:conclusioni}
	%limiti (su cifrature /qr-code / numero pagine) e sviluppi futuri
Nel corso di questo progetto, si è sviluppato un prototipo funzionante dell'applicazione richiesta completa di tutte le principali funzionalità. Tuttavia ancora molto si potrebbe fare per migliorarla e per superare le limitazioni che tuttora sussistono. Tra queste ci sono limitazioni riguardanti per esempio il numero di pagine del documento prodotto che non possono essere superiori a $1$. Un altro problema da superare è la limitata quantità di QR-Code che si possono inserire nella parte sottostante del documento.

Quindi tra gli sviluppi futuri di questo progetto ci sono:
\begin{itemize}
	\item la possibilità di produrre un documento di più pagine;
	\item la possibilità di inserire un maggior numero di QR-Code o di utilizzare questi ultimi al massimo delle loro possibilità per memorizzare quante più informazioni cifrate possibili;
	\item la possibilità di inserire immagini nel corpo del documento;
	\item migliorare l'affidabilità della funzione di perceptual hashing implementata conducendo studi di natura più avanzata;
	\item implementare un'applicazione mobile o un interfaccia web che renda più agevole l'uso del servizio (sfruttando magari la realtà aumentata nel caso mobile\ldots).
\end{itemize}